\documentclass[xcolor=dvipsnames, 9pt]{beamer}

\input{../tex/workshop_style.tex}

\title{NetworkX: Hacking Social Networks with Python}
\author{Drew Conway and Aric Hagberg}
%\institute{\includegraphics[width = 4cm]{500px-NYU_logo.png}}
\date{June 29, 2010}

\begin{document}
\begin{frame}[plain]
\titlepage
\end{frame}

\begin{frame}
\frametitle{Hacking}

\begin{columns}
\begin{column}{0.7\columnwidth}
\LARGE
Hacker

      A person who delights in having an intimate understanding of the
      internal workings of a system, computers and computer networks in
      particular.\footnote{
Internet Users' Glossary (Request for Comments 1392), January 1993.
http://www.rfc-editor.org/rfc/rfc1392.txt}
  
\end{column}
\end{columns}


\end{frame}

\begin{frame}
\frametitle{Overview}

\begin{description}

\item[1400-1430] Introduction to NetworkX (Hagberg)

\item[1430-1500]  Why do SNA with NetworkX (Conway)

\item[1500-1530] Getting started (Hagberg)

\item[1530-1545] Break

\item[1545-1645]  Basic Analysis (SNA) Conway

\item[1645-1715] Developing new algorithms (Hagberg)

\item[1715-????] Bring your own data/projects (Conway, Hagberg)

\end{description}

\end{frame}

\end{document}
