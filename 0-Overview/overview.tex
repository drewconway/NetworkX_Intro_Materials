\documentclass[xcolor=dvipsnames, 9pt]{beamer}

\newenvironment{code}{\begin{semiverbatim} \begin{footnotesize}}
{\end{footnotesize}\end{semiverbatim}}

\usepackage{graphicx}
\usepackage{amssymb}
\usepackage{amsfonts}
\usepackage{amsmath}
\usepackage{hyperref}
\usepackage{natbib}
\usepackage{color}
\usepackage{pdfsync}
\usepackage{chancery}
\usepackage{movie15}
\usepackage{pgfpages}
\usepackage{fancyvrb}
\usepackage{colortbl}
\usepackage{multirow}

% \definecolor{white}{rgb}{255,255,255}
% \definecolor{darkred}{rgb}{0.5,0,0}
% \definecolor{darkgreen}{rgb}{0,0.5,0}
% \definecolor{lightblue}{rgb}{0,0,0.7}

% \hypersetup{colorlinks,
%   linkcolor=white,
%   filecolor=darkred,
%   urlcolor=lightblue,
%   citecolor=darkblue}

\usepackage{beamerthemesplit}
\usetheme{Copenhagen}
\usecolortheme[named=Violet]{structure} 
\setbeamertemplate{navigation symbols}{}
\setbeamertemplate{itemize items}[triangle]
\setbeamertemplate{enumerate items}[default]
%\setbeameroption{show notes on second screen}
% \logo{\includegraphics[width = 2cm]{../images/logos/500px-NYU_logo.png}}

\newcommand{\R}{\mathbb{R}}
\renewcommand{\d}{\mathsf{d}}
\newcommand{\dd}{\partial}
\newcommand{\E}{\mathsf{E}}
\newcommand{\bb}{\mathbf}

\graphicspath{{../images/figures/}{../images/logos/}{../images/graphs}/}

\title{NetworkX: Hacking Social Networks with Python}
\author{Drew Conway and Aric Hagberg}
%\institute{\includegraphics[width = 4cm]{500px-NYU_logo.png}}
\date{June 29, 2010}

\begin{document}
\begin{frame}[plain]
\titlepage
\end{frame}

\begin{frame}
\frametitle{Hacking}

\begin{columns}
\begin{column}{0.7\columnwidth}
\LARGE
Hacker

      A person who delights in having an intimate understanding of the
      internal workings of a system, computers and computer networks in
      particular.\footnote{
Internet Users' Glossary (Request for Comments 1392), January 1993.
http://www.rfc-editor.org/rfc/rfc1392.txt}
  
\end{column}
\end{columns}


\end{frame}

\begin{frame}
\frametitle{Overview}

\begin{description}

\item[1400-1430] Introduction to NetworkX (Hagberg)

\item[1430-1500]  Why do SNA with NetworkX (Conway)

\item[1500-1530] Getting started (Hagberg)

\item[1530-1545] Break

\item[1545-1645]  Basic Analysis (SNA) Conway

\item[1645-1715] Developing new algorithms (Hagberg)

\item[1715-????] Bring your own data/projects (Conway, Hagberg)

\end{description}

\end{frame}

\end{document}
