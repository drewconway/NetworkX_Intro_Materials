\documentclass[xcolor=dvipsnames, 9pt]{beamer}
\input{../tex/workshop_style.tex}

\title{5 - Developing Algorithms}
\author{Drew Conway and Aric Hagberg}
%\institute{\includegraphics[width = 4cm]{500px-NYU_logo.png}}
\date{February 8, 2011}

\begin{document}
\begin{frame}[plain]
\titlepage
\end{frame}

\begin{frame}
\frametitle{Outline}
\begin{itemize}
\item Examples of some simple algorithms
\item Writing a new algorithm
\end{itemize}
\end{frame}

\begin{frame}[fragile]
\frametitle{Feature: Compact code - building new generators}
\centerline{\includegraphics[width=0.7\columnwidth]{model_title}}
\centerline{\includegraphics[width=0.35\columnwidth]{model}}
\end{frame}


\begin{frame}[fragile]
\frametitle{Feature: Compact code - building new generators}
\begin{block}{}
\tiny
\begin{lstlisting}
import bisect
import random
from networkx import MultiDiGraph

def scale_free_graph(n, alpha=0.41,beta=0.54,delta_in=0.2,delta_out=0):
    def _choose_node(G,distribution,delta):
        cumsum=0.0
        psum=float(sum(distribution.values()))+float(delta)*len(distribution)
        r=random.random()
        for i in range(0,len(distribution)):
            cumsum+=(distribution[i]+delta)/psum
            if r < cumsum:  
                break
        return i

    G=MultiDiGraph()
    G.add_edges_from([(0,1),(1,2),(2,0)])
    gamma=1-alpha-beta

    while len(G)<n:
        r = random.random()
        if r < alpha:
            v = len(G) 
            w = _choose_node(G, G.in_degree(),delta_in)
        elif r < alpha+beta:
            v = _choose_node(G, G.out_degree(),delta_out)
            w = _choose_node(G, G.in_degree(),delta_in)
        else:
            v = _choose_node(G, G.out_degree(),delta_out)
            w = len(G) 
        G.add_edge(v,w)
    return G

\end{lstlisting}
\end{block}
\end{frame}


\begin{frame}[fragile]
\frametitle{Feature: Python expressivity - a simple algorithm}
Python is easy to write and read
\begin{block}{Breadth First Search}
\begin{lstlisting}
from collections import deque

def breadth_first_search(g, source):
    queue = deque([(None, source)])
    enqueued =  set([source])
    while queue:
        parent, n = queue.popleft()
        yield parent, n
        new = set(g[n]) - enqueued
        enqueued |= new
        queue.extend([(n, child) for child in new])
\end{lstlisting}
\end{block}
Credit: Matteo Dell'Amico
\end{frame}


\begin{frame}
\frametitle{Degree centrality}
\begin{columns}
\begin{column}{0.5\textwidth}
For a graph with $n$ nodes
\begin{equation*}
C_D(v) = \frac{deg(v)}{n-1}
\end{equation*}
\centerline{\includegraphics[width=1.0\columnwidth]{star}}
\end{column}
\begin{column}{0.5\textwidth}

\lstinputlisting[firstline=1]{code/degree_centrality.py.doctest}

\end{column}
\end{columns}
\end{frame}


\begin{frame}
\frametitle{Degree centrality 1}
\lstset{showspaces=true,
showtabs=true,
tab=\rightarrowfill}
\lstinputlisting{code/degree_centrality1.py}
\end{frame}


\begin{frame}
\frametitle{Degree centrality 2}
\lstinputlisting{code/degree_centrality2.py}
\end{frame}


\begin{frame}
\frametitle{Degree centrality 3}
\lstinputlisting{code/degree_centrality3.py}
\end{frame}


\begin{frame}
\frametitle{Degree centrality 4}
\lstinputlisting[lastline=16]{code/degree_centrality4.py}
\end{frame}


\begin{frame}
\tiny
\frametitle{Degree centrality in NetworkX}
\lstinputlisting{code/nx_degree_centrality.py}
\end{frame}


\begin{frame}
\frametitle{Degree centrality 5}
This algorithm is really a one-liner:
\lstinputlisting{code/degree_centrality5.py}
\end{frame}

\begin{frame}
\frametitle{Design Ego Graph}
\centerline{Create network of neighbors centered at node n}
\centerline{\includegraphics[width=0.7\columnwidth]{1311774-2c}}
\end{frame}


\begin{frame}
\frametitle{Ego Graph: getting started}
\begin{columns}
\begin{column}{0.6\textwidth}
\small
\lstinputlisting{code/ego_graph.py}
\end{column}
\begin{column}{0.4\textwidth}
\centerline{\includegraphics[width=0.7\columnwidth]{starfish.png}}
\centerline{\includegraphics[width=0.7\columnwidth]{star.png}}

Hints:
\begin{itemize}
\item only condsider Graph()
\item use G.neighbors(v) 
\item similar as G[v]
\item don't worry about attributes
\end{itemize}

\end{column}
\end{columns}
\url{http://bit.ly/cKJhic}
%\footnotesize{
%\url{http://github.com/drewconway/NetworkX_Intro_Materials/raw/master/5-Developing-Algorithms/code/ego_graph.py}}
\end{frame}

\begin{frame}
\frametitle{Ego Network}

My solution

\end{frame}

\end{document}
