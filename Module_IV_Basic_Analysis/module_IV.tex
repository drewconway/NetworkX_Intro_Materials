%
%  module_IV.tex
%
%  Created by Drew Conway on 2010-06-01.
% 
%
\documentclass[xcolor=dvipsnames, 9pt]{beamer}

\newenvironment{code}{\begin{semiverbatim} \begin{footnotesize}}
{\end{footnotesize}\end{semiverbatim}}

\usepackage{graphicx}
\usepackage{amssymb}
\usepackage{amsfonts}
\usepackage{amsmath}
\usepackage{hyperref}
\usepackage{natbib}
\usepackage{color}
\usepackage{pdfsync}
\usepackage{chancery}
\usepackage{movie15}
\usepackage{pgfpages}
\usepackage{fancyvrb}
\usepackage{colortbl}

% \definecolor{white}{rgb}{255,255,255}
% \definecolor{darkred}{rgb}{0.5,0,0}
% \definecolor{darkgreen}{rgb}{0,0.5,0}
% \definecolor{lightblue}{rgb}{0,0,0.7}

% \hypersetup{colorlinks,
%   linkcolor=white,
%   filecolor=darkred,
%   urlcolor=lightblue,
%   citecolor=darkblue}

\usepackage{beamerthemesplit}
\usetheme{Copenhagen}
\usecolortheme[named=Violet]{structure} 
\setbeamertemplate{navigation symbols}{}
\setbeamertemplate{itemize items}[triangle]
\setbeamertemplate{enumerate items}[default]
%\setbeameroption{show notes on second screen}
%\logo{\includegraphics[width = 2cm]{nyulogo.png}}

\newcommand{\R}{\mathbb{R}}
\renewcommand{\d}{\mathsf{d}}
\newcommand{\dd}{\partial}
\newcommand{\E}{\mathsf{E}}
\newcommand{\bb}{\mathbf}

\title{Module II - Basic Analysis}
\author{Drew Conway --- Department of Politics}
\institute{\includegraphics[width = 4cm]{../images/logos/500px-NYU_logo.png}}
\date{June 29, 2010}

\begin{document} 

\begin{frame}[plain]
  \titlepage  
\end{frame}

\begin{frame}
	\frametitle{Agenda for Module IV}
	Loading data from multiple sources
	\begin{itemize}
	   \item Local network data files
	   \item Connecting to a database
	   \item Building directly from the Internet
	\end{itemize}
	\uncover<2->{Brief review of Python \texttt{dict} data type
	\begin{itemize}
	   \item Why it is so useful
	   \item How \texttt{NetworkX} utilizes it
	\end{itemize}}
	\uncover<3->{Running basic centralities
	\begin{itemize}
	   \item Degree, Closeness, Betweeness Eigenvector
	   \item Calculating degree distribution
	   \item Plotting statistics using \texttt{matplotlib}
	   \item Calculating cliques, clustering and transitivity
	\end{itemize}}
	\uncover<4->{Outputting data into multiple formats
	\begin{itemize}
	   \item Writing network data
	   \item Saving network analysis statistics
	\end{itemize}}
	\uncover<5->{Basic visualization
	\begin{itemize}
	   \item Review of \texttt{NetworkX}'s plotting algorithms
	   \item Adding analysis to visualization 
	\end{itemize}}
\end{frame}

\section{Loading data from multiple sources} % (fold)
\label{sec:loading_data_from_multiple_sources}

\subsection{Local network data} % (fold)
\label{sub:local_network_data}

\begin{frame}[fragile]
    \frametitle{Loading a network file}
    As we have seen, one of the main advantages of working with \texttt{NetworkX} is that it can read many different network formats
    \begin{itemize}
        \item For those that are unfamiliar with working at the \textbf{command-line}, however, the process can be confusing
    \end{itemize}
    \begin{block}{NX syntax for loading a file}
        \begin{tabular}{cccc}
        \alert<2>{$>>> G$} & = & \alert<3>{read\_format(``path/to/file.txt''}, & \alert<4>{\emph{...options...}}) \\
        \alert<2>{\uparrow} & & \alert<3>{\uparrow} & \alert<4>{\uparrow} \\
        \alert<2>{\scriptsize{Net variable}} & & \alert<3>{\scriptsize{NX function, file directory path}} & \alert<4>{\scriptsize{Graph type, nodes type, etc.}}
        \end{tabular}
    \end{block}
    \uncover<5->{Let's try!
    \begin{itemize}
        \item We will load the edge list of Hartford drug users network
        \item Specify that the network be a directed graph, and the nodes be integers
        \item Use \texttt{info()} to check that data has been loaded correctly
    \end{itemize}}
    \begin{center}
        \uncover<6->{\alert<6>{It's time to fire up your console and load Python!}}
    \end{center}
\end{frame}

\begin{frame}[fragile]
    \frametitle{Loading the Hartford drug users network}
    \begin{block}{Starting \texttt{NetworkX} and loading data}
        \begin{code}
\scriptsize{>>> from networkx import *
>>> hartford=\alert<3>{read_edgelist}("\alert<4>{../../data/hartford_drug.txt}",\alert<5>{create_using=DiGraph()},\alert<6>{nodetype=int})
>>> \alert<7>{info(hartford)}
Name:                  
Type:                  DiGraph
Number of nodes:       212
Number of edges:       337
Average in degree:     1.5896
Average out degree:    1.5896}
        \end{code}
    \end{block}
\uncover<2->{What did we just do?}
\begin{itemize}
    \item \uncover<3->{Used the \texttt{read\_edgelist} function to load EL file}
    \item \uncover<4->{Specified path to Hartford drug users file}
    \item \uncover<5->{Used the \texttt{create\_using} option to force NX to create as a directed graph}
    \item \uncover<6->{Used the \texttt{nodetype} option to force NX to store nodes as integers}
    \item \uncover<7->{Used the \texttt{info} function to check that it all worked}
\end{itemize}
\uncover<8->{Some formats may have more or less options, \textbf{always check the documentations!}}
\end{frame}

% subsection local_network_data (end)

\subsection{Connecting to a database} % (fold)
\label{sub:connecting_to_a_database}

\begin{frame}[fragile]
    \frametitle{Building a network from a database}
    
\end{frame}


% subsection connecting_to_a_database (end)

% section loading_data_from_multiple_sources (end)

\end{document}
